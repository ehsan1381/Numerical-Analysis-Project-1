\documentclass[12pt, letterpaper]{article}

\usepackage{amsmath}


% load persian language and select font
\usepackage{xepersian}
\settextfont{BNazanin}

\title{گزارش کار پروژه اول آنالیز عددی 1}
\author{گروه 6}

\begin{document}
\maketitle

\section{سوال 1}
\subsection{صورت سوال}
فرض کنید \(fl(y)\) عدد \(k\) رقمی قطع شده \(y\) باشد، نشان دهید:

\[\frac{\big| y-fl(y) \big| }{\big| y \big|} \le 10^{-k+}\]

\subsection{پاسخ}

% answer to question 1


% leave space between question 1 and 2
\vspace{5mm}

% start of question 2
\section{سوال 2}
\subsection{صورت سوال}
فرض کنید
\(f(x) = \frac{2 \cdot log(1+x) \: + \: 2 \cdot i tan^{-1}(ix) \: + \: x^2}{-x^4}\)
و 
\(\lim_{x\to 0}\frac{f(x)}{x^p} = C \ne 0\)
با داشتن سری مکلورن توابع 
\(log(x)\)
و 
\(tan^{-1}(x)\)
مقادیر 
\(C\)
و 
\(p\)
را بیابید.

\subsection{پاسخ}
همانطور که در فایل میپل حل این سوال قابل مشاهده است، در مرحله اول با تعریف ظابطه اصلی تابع و رسم نمودار آن سعی میکنیم درکی هندسی از رفتار آن بیابیم. نمودار اول نشان میدهد که تابع در 0 به مقداری نزدیک به 0 میل میکند. اما با برسی دقیق تر و محدود کردن دامنه نمایش نمودار مشاهده میشود که تابع در مقادیر نزدیک به صفر شدیدا نوسان میکند. \\
برای حل سوال در اپتدا بسط مکلورن توابع را تا درجه 10 محاسبه میکنیم.

\end{document}























